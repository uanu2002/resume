% !TEX program = xelatex
\documentclass{resume}
\usepackage{zh_CN-Adobefonts_external} % Simplified Chinese Support using external fonts (./fonts/zh_CN-Adobe/)
\usepackage{lastpage}
\usepackage{fancyhdr}
\usepackage{linespacing_fix}
\renewcommand{\labelitemii}{$\circ$}
 
\begin{document}

\name{吴家焱}
\basicInfo{
  % \phone{} \textperiodcentered\
  \email{roifewu AT gmail DOT com} \textperiodcentered\
  \github[roife]{https://github.com/roife} \textperiodcentered\
  \homepage[roife.github.io]{https://roife.github.io}
}

\section{教育背景}
\datedsubsection{\textbf{北京航空航天大学}}{2019.09 -- 现在}
\role{本科,在读}{计算机科学与技术专业; GPA 3.83/4.00; 排名 19/205}

\section{科研经历}
\datedsubsection{\textbf{北京航空航天大学}}{2021.08 -- 2022.06}
\begin{itemize}
  \item 面向 GPU 统一虚拟内存的数据管理优化技术:通过识别应用的数据管理瓶颈,实现异构内存下数据的优化放置
  \item LLVM-IR 到 ARMv7 的轻量编译器,包括 C++ 实现的优化器和 LLVM 后端裁剪两部分
\end{itemize}

\section{项目开发}
\subsection{\textbf{底层系统}}
\begin{itemize}
  \item \textbf{Ayame 编译器} (Java, ARM):SysY 编译器,可导出 LLVM-IR/ARM32 汇编,毕昇杯第二名(合作)

  \item \textbf{Racoon 编译器} (Rust, LLVM-IR):
        SysY 到 LLVM IR 编译器;北航软院 19 级编译课设 Rust 版参考实现

  \item \textbf{MIPS 处理器} (Verilog, MIPS):
        五级流水线 CPU,包含协处理器 CP0 子集的实现,支持多种中断异常

  \item \textbf{MOS 操作系统} (C, MIPS):
        操作系统内核,包含了一个较完善的 Shell
\end{itemize}

\subsection{\textbf{应用开发}}
\begin{itemize}
  \item \textbf{航概} (SwiftUI, Vue, Rails):
        《航空航天概论》刷题应用,支持 Web 与 iOS;应用「航概」已上架 AppStore(合作)

  \item \textbf{求职岛} (微信小程序, Vue, Django):
        校内实验室招聘平台,支持微信小程序与 Web 端(合作)
\end{itemize}

\section{获奖情况}
\subsection{\textbf{竞赛}}
\begin{itemize}
  \item \datedline{蓝桥杯软件类省赛(\textit{一等奖})/蓝桥杯软件类国赛(\textit{三等奖})}{2021.06}
  \item \datedline{全国大学生计算机系统能力大赛编译系统设计赛·华为毕昇杯(\textit{一等奖,第二名})}{2021.08}
  \item \datedline{第 24 届CCF-CSP 软件能力认证(\textit{单次排名 1.51\%,累计排名 0.99\%})}{2021.09}
  \item \datedline{第三十二届“冯如杯”竞赛主赛道制作组(三等奖)}{2022.06}
\end{itemize}

\subsection{\textbf{奖学金}}
\begin{itemize}
  \item \datedline{北航本科生学习优秀奖学金(两次\textit{二等奖})}{2020.11,2021.11}
  \item \datedline{北航本科生学科竞赛奖学金(\textit{特等奖})}{2021.11}
\end{itemize}

\section{专业技能}
\begin{itemize}
  \item \textbf{编程语言}:C, C++, Java, Rust, Swift, Python, JavaScript, Ruby, Verilog, Coq

  \item \textbf{编译/程序语言理论}:
    \begin{itemize}
      \item 熟悉 ANTLR 等解析器生成器,了解 SSA 和相关优化算法,了解过 LLVM
      \item 了解函数式语言和类型系统,理解基于 Dependent Type 的形式化验证技术并学习过 Coq
    \end{itemize}

  \item \textbf{应用开发}:
    \begin{itemize}
      \item 技术栈:Web 前端 (Vue),Web 后端 (Rails, Django),iOS (SwiftUI)
      \item 掌握 Docker, CI, Redis, PostgreSQL 等工具的使用
    \end{itemize}

  \item \textbf{开发环境}:常在 Linux 下使用 Emacs 与 JetBrains IDE;熟悉 Git 与 Gitflow 协作流程
\end{itemize}
 
\section{其他}
\begin{itemize}
  \item \textbf{助教工作}:
        \begin{itemize}
          \item \datedline{\textbf{程序设计基础训练}(北航信息类专业)}{2020.09 -- 2021.02}
                \role{课程助教}{负责出题和答疑}
          \item \datedline{\textbf{面向对象设计与构造}(北航计算机学院 S.T.A.R 教辅团队)}{2021.08 -- 2022.06}
                \role{课程助教 \& 系统组}{参与课程系统(Rails \& GitLab)的开发与日常运维,课程出题、答疑、跟班等}
        \end{itemize}
  \item \textbf{社会实践}:志愿服务总时长超 200 小时;担任北航微软俱乐部的技术部副部长
  \item \textbf{博客}:已在 \url{https://roife.github.io} 创作约 170 篇文章,月访问量最高逾 1.5k
  \item \textbf{外语}:英语(CET-6)
\end{itemize}
\end{document}
