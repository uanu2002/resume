% !TEX program = xelatex
\documentclass{resume}
\usepackage{zh_CN-Adobefonts_external} % Simplified Chinese Support using external fonts (./fonts/zh_CN-Adobe/)
\usepackage{lastpage}
\usepackage{fancyhdr}
\usepackage{linespacing_fix}
\renewcommand{\labelitemii}{$\circ$}

\begin{document}

\name{吴建宇}

\section{个人信息}
\begin{itemize}
\item{姓名}:吴建宇
\item{性别}:男
\item{学号}:21371377
\item{微信}:uanuwx
\item \phone{15050772492}
\item \email{uanu2002@163.com}
\item \github[uanu2002]{https://github.com/uanu2002}
\item \homepage[uanu2002.top]{https://www.uanu2002.top}
\end{itemize}

\section{教育背景}
\datedsubsection{\textbf{北京航空航天大学}}{2021.09 -- 现在}
\role{本科大二,在读}{人工智能专业; GPA 3.85/4.00; 排名 5/30}
\subsection{\textbf{课程成绩}}
\begin{itemize}
  \item \textbf{程序设计基础}:\textbf{93}
  \item \textbf{数据结构与程序设计(信息类)}:\textbf{99}
\end{itemize}

\section{获奖情况}
\subsection{\textbf{竞赛}}
\begin{itemize}
  \item \datedline{蓝桥杯软件类省赛(\textit{二等奖})}{2022.06}
\end{itemize}

\section{项目开发}
\subsection{\textbf{网页开发}}
\begin{itemize}
  \item \textbf{你我牠} (Html, JavaScript, CSS):
        作为队长和主程序员进行网站开发。
\end{itemize}

\section{专业技能}
\begin{itemize}
  \item \textbf{编程语言}:C, C++, Java, Python, JavaScript
  \item \textbf{开发环境}:能熟练使用Linux; 熟悉 Git 与 Gitflow 协作流程
\end{itemize}

\section{申请原因及优势}
\begin{itemize}
  \item \textbf{申请原因}:大一新生第一次接触程序设计,几乎都需要花费大量的时间入门和debug。我想利用我的知识和经验让学弟学妹们不再踩坑,不再因debug而痛苦,从而更快更好的掌握程序设计这门课。
  \item \textbf{个人优势}
    \begin{itemize}
      \item \textbf{基础扎实}:对大一学年程序设计相关课程透彻且取得较好成绩;写代码和debug经验丰富,擅长分析和解决bug。
	 \item \textbf{学习创新}:乐于学习新知识、掌握新技能。担任助教后我会积极参与OJ题目和题解的编写。
	 \item \textbf{善于分析总结}:对知识点有较强的总结能力,同时可以以博客形式进行分享,已经在个人博客\url{uanu2002.top}上创作数十篇文章。
	 \item \textbf{乐于助人}:平时善于与他人沟通并乐于帮助他人解决问题。担任助教后我能够腾出足够的时间为学弟学妹们解答疑惑,此外还会定期或在考期前举办串讲。
    \end{itemize}
\end{itemize}
\end{document}